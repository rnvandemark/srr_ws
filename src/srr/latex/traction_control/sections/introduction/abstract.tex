Space exploration continues to become more accessible as technology develops and international interest for it is reignited. Observing Mars is one of the highest priorities at the present and in the near future for a number of reasons, including the fact that it is the closest to us, it may have once had life, has been confirmed to have had liquid water at some point, and decades of intimate exploration has established a solid foundation for a human presence on the "Red Planet". There are many challenges pertaining to exploring unknown territory over long distances. NASA's approach to exploring and researching Mars was through the \ac{SRR}, the first iteration of which was created in 1997 by the JPL Jet Propulsion Laboratory, and the most recent being launched this year in 2020. The scope of this project focuses on an emergent implementation for a type of traction control algorithm, designed for the \ac{CMR}, as damage to the wheels was developing at an incredibly alarming rate due to the fact that the wheels were slipping over rough, elevated surfaces. This report offers a detailed description of the original problem, the algorithm proposed and implemented as the solution, and discussion about an attempt at simulating the algorithm on the \ac{SRR}.
