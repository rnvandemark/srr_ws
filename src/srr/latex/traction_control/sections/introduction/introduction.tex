The Mars Rover is a Sample Return Rover designed by the JPL NASA Laboratory. It consists of a chassis with a central revolute joint on both the left and right sides of the robot. The central revolute joint is connected to two links (on each side),  for the front and rear wheels. Each respective link is connected to another link downward that is connected to a wheel. The front links are revolute so that the robot can turn left and right. Note that the original paper is based off of the Curiosity Rover, which has six wheels. The focus on this paper will be on the Sample Return Rover, which has four wheels. 

Primitive designs of this robot had a large amount of damage to the rover’s wheels. This wheel damage reduced the longevity of the Mars Rover mission by a great amount. NASA had to counteract this damage through researching the cause of the wheel damage. The rover did not properly avoid terrain obstacles nor did it deal with traction loss. The goal for this project is to recreate the traction control algorithm and simulate the results. 

The algorithm for the traction control is a velocity-based algorithm. One wheel on the rover will be rotating much faster than the others. The traction control system then applies a brake to that wheel to reduce its slip and then reducing wheel slip.
NEED TO FIX !