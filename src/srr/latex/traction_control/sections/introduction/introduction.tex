Here is some example text, which lives in the file \textit{srr\textunderscore ws/src/srr/latex/ traction\textunderscore control/sections/introduction/introduction.tex}, and is made an input in \textit{srr\textunderscore ws/src/srr/latex/traction\textunderscore control/traction\textunderscore control.tex}. LaTeX uses a bunch of macros to stylize and simplify writing reports like these, like how the filepaths were italicized with the \textbf{\textbackslash textit} command. There's a text environment, where this text is being written in, and then there's a math environment, which can format stuff to be all pretty such as:

\begin{equation}\label{traction_control:intro:quad}
	x_{1,2} = \frac{-b \pm \sqrt{b^{2} - 4ac}}{2a}
\end{equation}

And then you can reference the super important equation \eqref{traction_control:intro:quad} by naming it, so you never have to keep track of which number equation it is, etc. \\

Ending a line with \text{\textbackslash \textbackslash} will force a new line, like just above this sentence. Also, here's a figure with a centered caption below it:

\begin{figure}[htbp]
	\centering
	\includegraphics[width=.9\textwidth]{sections/introduction/images/srr.png}
	\caption{An Early Version of the Sample-Return Rover}
\end{figure}

Again, it's numbered automatically, so we don't have to change that if we decide to move stuff around. If we decide to change the style later, we just have to renew commands and recompile, we don't have to change the content of these files, etc. There are even likely premade IEEE stylesheets that can be imported and automatically applied to everything in this document. \\

Press the green button at the top to compile the main .tex file, TeXstudio will render a preview of the document for you. You can right click on the PDF preview and click "Go to Source" if you want to open the location of whatever it is you clicked on, if you see a typo you want to change quickly, etc. Try it with this paragraph or the figure.
