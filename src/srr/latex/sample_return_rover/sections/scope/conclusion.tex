Designing a cad model without much reference was quite difficult to get the dimensions of the robot to look right, some parts had to be redesigned. The fallback goal is to have the robot and arm operating in separate gazebo environments so that the controls algorithm can be tested independently, and this was successfully implemented. A lot of the concepts learnt from class were used for this project. Knowing how to do kinematics of a robot was very beneficial for calculating kinematics of the arm. Even though that the arm is four degrees of freedom, the arm would not properly turn in Gazebo because it needed dummy frames. As a result, writing out and assigning dummy frames fixed the issue. The robot was successfully created in SolidWorks and exported to a URDF. The controls were tuned accordingly in Gazebo so that the rover can be operated via tele-op. Calculations were determined and scripted. Overall a lot was learned from this project and it was a good experience.
